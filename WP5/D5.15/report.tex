\documentclass{deliverablereport}

\deliverable{hpc}{GAP-HPC-report}
\deliverydate{XX/YY/201Z}
\duedate{31/08/2019 (M48)}
\author{Author names}

\begin{document}
\maketitle
% This will be the abstract, fetched from the github description
\githubissuedescription

% write the report here

% Original list of sections and subsections created from
% https://github.com/OpenDreamKit/OpenDreamKit/issues/113

\section{Developments in the core GAP system}

\subsection{libGAP: allowing 3rd party code to link GAP as a library}

GAP 4.10.0 (November 2018) is the first official GAP release
which provided an experimental way to allow 3rd party code to 
link GAP as a library. It was based on the libGAP code by SageMath, 
but different: while we aim to provide the same functionality, 
we do not rename any symbols, and we do not provide the same API. 

Since then, we improved the robustness of our libGAP implementation
in GAP 4.10.1 (February 2019) and GAP 4.10.2 (June 2019) releases,
and extended its API with new functionality (See Chapter 2 
``Changes between GAP 4.9 and GAP 4.10'' of the 
``GAP - Changes from Earlier Versions'' manual for the detailed
descriptions). This allowed SageMath to drop its custom 
modifications for GAP and use the official, documented and regularly 
tested GAP interface instead.
%
%\subsection{other changes}
%
%see releases overview at https://www.gap-system.org/Manuals/doc/changes/chap0.html)
%
%\section{HPC-GAP}
%
%merging into mainstream GAP releases (see here)
%
%meataxe64 interface (https://github.com/gap-packages/meataxe64 and https://meataxe64.wordpress.com/)
%
%\section{Packages}
%
%package manager
%
%improving the health of the package ecosystem
%
%\section{Jupyter and derivatives}
%
%JupyterKernel
%
%Francy
%
%JupyterViz
%
%other packages by @nathancarter
%
%\section{GAP distributions}
%
%Improved testing
%
%Docker and other alternatives
%
%\section{Events}
%
%community building events
%
%training events, Software Carpentry lesson, other workshops
%
%\section{Interfaces, WP6}
%
%high-level interoperability
%
%persistent memoisation
%
%\section{A collection of demonstrators}
%
%"full-stack semigroups"
%
%persistent memoisation
%
%databases


\end{document}

%%% Local Variables:
%%% mode: latex
%%% TeX-master: t
%%% End:

